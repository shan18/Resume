\cvsection{Projects}
\begin{cventries}
    \cventry
    {E-commerce website built using Django | \href{https://shan-kart.herokuapp.com/}{\textcolor{Blue}{Website}} | \href{https://github.com/shan18/Kart}{\textcolor{Blue}{GitHub}}}
    {Kart}
    {}
    {\textbf{Dec. 2017 - Jan. 2018}}
    {
        \begin{cvitems}
          \item{Built the backend entirely on Django. Used jQuery in some places to make the website asynchronous.}
          \item{Used signals, custom model managers, and custom querysets extensively to keep most of the code logic within the models and to make the communication between the linked models effective.}
          \item{\textbf{Tools}: Python, Django, Bootstrap, jQuery, Ajax, jsrender, chart.js}
          \item{\textbf{Services}: stripe, mailchimp, Amazon Web Services, heroku, sendgrid}
        \end{cvitems}
    }
    \cventry
    {Machine Learning, Natural Language Processing | \href{https://github.com/shan18/Autoranking-Amazon-Reviews}{\textcolor{Blue}{GitHub}}}
    {Autoranking Amazon Reviews}
    {}
    {\textbf{Oct. 2017}}
    {
      \begin{cvitems}
        \item{Ranking the reviews on Amazon according to their helpfulness score.}
        \item{The problem was modeled as a regression problem. The performance was evaluated by using the coefficient of determination and rank correlation.}
        \item{Predictions were made based on various categories of features of the review text, and other metadata associated with the review, with the purpose of generating a rank for a given list of reviews.}
        \item{\textbf{Tools}: Python, Numpy, Pandas, textblob, scikit-learn}
      \end{cvitems}
    }
    \cventry
    {Android Application | \href{https://play.google.com/store/apps/details?id=com.nitmz.morphosis&hl=en}{\textcolor{Blue}{Google Play Store}} | \href{https://github.com/morphosis-nitmz/Morphosis-2k17-Android}{\textcolor{Blue}{GitHub}}}
    {Morphosis, NIT Mizoram}
    {NIT Mizoram}
    {\textbf{Mar. 2017 - Apr. 2017}}
    {
      \begin{cvitems}
        \item {Android App for the annual technical fest of NIT Mizoram}
        \item{Contains all the information of various events to be conducted during the technical fest.}
        \item{Contains a game called Scooby Dooby Doo which gives live leaderboard updates.}
        \item{\textbf{Tools}: Java, Android Studio, Firebase}
      \end{cvitems}
    }
\end{cventries}  

