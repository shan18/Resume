%%%%%%%%%%%%%%%%%%%%%%%%%%%%%%%%%%%%%%
%     Projects
%%%%%%%%%%%%%%%%%%%%%%%%%%%%%%%%%%%%%%

\section{Projects}

\runsubsection{TensorNet}
\descript{| Deep Learning | \textcolor{Blue}{\href{https://github.com/shan18/TensorNet}{Github}} | \textcolor{Blue}{\href{https://pypi.org/project/torch-tensornet/}{PyPI}}}
\location{Mar 2020 - Present}
\begin{tightemize}
\item Developed a high-level deep learning library on top of PyTorch.
\item Used modularization and OOP concepts extensively to maintain a clean and understandable codebase.
\item \custombold{Tools:} Python, PyTorch, NumPy, Pillow.
\end{tightemize}
\sectionsep

\runsubsection{Topic Based Image Captioning}
\descript{| Deep Learning | \textcolor{Blue}{\href{https://github.com/shan18/Topic-Based-Image-Captioning}{Github}}}
\location{Oct 2018 - May 2019}
\begin{tightemize}
\item Developed a novel image captioning model using CNNs and LSTMs.
\item Created a system where LSTMs were given additional information (topics) extracted from image captions using Latent Dirichlet Allocation (LDA).
\item \custombold{Tools:} Python, Tensoflow-Keras, NLTK, OpenCV-Python, MSCOCO-2017 Dataset.
\end{tightemize}
\sectionsep
